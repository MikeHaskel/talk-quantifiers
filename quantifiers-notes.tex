\documentclass{article}
\usepackage{amsmath,amsthm,amssymb}
\usepackage{hyperref}

\theoremstyle{plain}
\newtheorem{thm}{Theorem}[section]
\newtheorem{prop}[thm]{Proposition}
\newtheorem{lemma}[thm]{Lemma}

\theoremstyle{definition}
\newtheorem{defn}[thm]{Definition}
\newtheorem{fact}[thm]{Fact}
\newtheorem{example}[thm]{Example}

\newcommand{\defterm}{\emph}
\newcommand{\ringsig}{\sigma_{\operatorname{rng}}}
\newcommand{\fieldthy}{T_{\operatorname{fld}}}
\newcommand{\acf}{T_{\operatorname{acf}}}
\newcommand{\tuple}{\bar}
\DeclareMathOperator{\Cn}{Cn}

\newcommand{\Rho}{\mathrm{P}}
\renewcommand{\phi}{\varphi}


\title{A Brief Taste of Quantifiers}
\author{Mike Haskel}
\date{August 22, 2015}


\begin{document}
\maketitle

\begin{abstract}
  These are notes for a talk given on August 22, 2015 as part of the
  University of Notre Dame's orientation for incoming math graduate
  students. The purpose of the talk is to introduce the new students
  to the material covered in the first year logic course, which
  primarily covers both model theory and computability. The unifying
  theme of the talk is first-order definability and quantifier
  complexity. In model theory, we discuss quantifier elimination in
  algebraically closed fields and prove the Ax-Grothendieck
  Theorem. In computability, we introduce the arithmetical hierarchy,
  discuss its relationship with computability in general, and prove
  that it does not collapse.
\end{abstract}

\section{Introduction}

\section{Model Theory}

\subsection{Basic Framework and Notation}

\begin{defn}
  A \defterm{signature} $\sigma$ is a collection of constant symbols,
  function symbols, and relation symbols, together with an
  \defterm{arity} for each function and relation symbol (that is, a
  natural number specifying how many arguments the symbol takes).
\end{defn}

\begin{defn}
  Given a signature $\sigma$, a \defterm{$\sigma$-structure} is a set
  and an interpretation for every symbol in $\sigma$. That is, a
  $\sigma$-structure $\mathcal{M}$ is a set $M$ together with
  \begin{itemize}
  \item an element $c^\mathcal{M} \in M$ for every constant symbol $c$,
  \item a function $f^\mathcal{M} : M^n \to M$ for every $n$-ary function symbol $f$, and
  \item a relation $R^\mathcal{M} \subseteq M^n$ for every $n$-ary relation symbol $R$.
  \end{itemize}
\end{defn}

As a notational convention, if $\mathcal{M}$ is a structure, $M$ will
always denote its underlying set.

\begin{example}
  The \defterm{signature of rings}, $\ringsig$, consists of constant
  symbols $0, 1$, binary operations $+, \cdot$, and a unary operation
  $-$. Every ring is naturally a $\ringsig$-structure.
\end{example}

Note that being a $\ringsig$-structure is actually much weaker than
being a ring: a $\ringsig$-structure need not satisfy any of the ring
axioms. To actually study rings with this framework, for example, we
need a little more.

\begin{defn}
  Given a signature $\sigma$ and a tuple of variable symbols
  $\tuple{x}$, a \defterm{$\sigma$-formula in $\tuple{x}$},
  $\phi(\tuple{x})$, is a sequence of symbols representing a
  first-order claim, with free variables from among $\tuple{x}$ and
  primitive symbols from $\sigma$.

  Terms in a formula are built as constant symbols, variable symbols,
  and application of function symbols to other terms (respecting
  arity, of course). The formulas themselves are built as $t_1 = t_2$
  (where $t_1, t_2$ are terms), application of relation symbols to
  terms (respecting arity), finite conjunctions, finite disjunctions,
  negations, implications, and universal and existential quantifiers.

  A \defterm{sentence} $\rho$ is a formula with no free variables.
\end{defn}

\begin{defn}
  Let $\sigma$ be a signature, $\mathcal{M}$ a $\sigma$-structure,
  $\phi(\tuple{x})$ a $\sigma$-formula, and $\tuple{a} \in M^n$ (where
  $\tuple{a}, \tuple{x}$ both have length $n$). Then write
  $\mathcal{M} \models \phi(\tuple{a})$ if the formula
  $\phi(\tuple{x})$ holds of $\tuple{a}$ in the obvious way, with
  constant, function, and relation symbols interpreted via
  $\mathcal{M}$, and quantifiers ranging over $M$. Write
  $\phi^\mathcal{M} = \{ \tuple{a} \in M^n \mid \mathcal{M} \models
  \phi(\tuple{a})\}$, the set defined in $\mathcal{M}$ by
  $\phi(\tuple{x})$.

  The double turnstyle symbol, $\models$, is pronounced ``models.''
\end{defn}

\begin{example}
  Let $\phi(x)$ be $\exists y: x \cdot y = 1$, a
  $\ringsig$-formula. If $\mathcal{M}$ is a ring and $a \in M$, the
  $\mathcal{M} \models \phi(a)$ if and only if $a$ has a
  multiplicative inverse.
\end{example}

\begin{example}
  Let $\rho$ be $\forall x: x = 0 \vee (\exists y: x \cdot y = 1)$, a
  $\ringsig$-sentence. If $\mathcal{M}$ is a ring, then $\mathcal{M}
  \models \rho$ if and only if $\mathcal{M}$ is furthermore a field.
\end{example}

\begin{defn}
  Let $\sigma$ be a signature and $\Rho$ set of $\sigma$-sentences.
  \begin{itemize}
  \item If $\mathcal{M}$ is a $\sigma$-structure, write $\mathcal{M}
    \models \Rho$ if $\mathcal{M} \models \rho$ for every $\rho \in
    \Rho$.
  \item If $\mathcal{\rho}$ is a $\sigma$-sentence (not necessarily in
    $\Rho$), write $\Rho \models \rho$ if, for all $\sigma$-structures
    $\mathcal{M}$ with $\mathcal{M} \models \Rho$, $\mathcal{M}
    \models \rho$.
  \end{itemize}
\end{defn}

\begin{defn}
  Let $\sigma$ be a signature and $\Rho$ a set of $\sigma$-sentences.
  \begin{itemize}
  \item Define $\Cn(\Rho)$, the \defterm{consequences} of $\Rho$, to
    be $$\{\rho \mid \Rho \models \rho\}\text{.}$$
  \item Say that $\Rho$ is \defterm{consistent} if it has a model,
    i.e., there is $\mathcal{M}$ such that $\mathcal{M} \models \Rho$.
  \item Say that $\Rho$ is a $\sigma$-\defterm{theory} if it is closed
    under consequence (i.e.\ $\Cn(\Rho) = \Rho$) and consistent.
  \end{itemize}
\end{defn}

By convention, theories are usually written $T$ instead of $\Rho$.

\begin{example}\label{example:fieldthy}
  The \defterm{theory of fields}, $\fieldthy$, is the set of consequences of
  the following $\ringsig$-sentences:
  \begin{itemize}
  \item $\forall x,y,z: x + (y + z) = (x + y) + z$
  \item $\forall x,y: x + y = y + x$
  \item $\forall x: x + 0 = x$
  \item $\forall x: x + (-x) = 0$
  \item $\forall x,y,z: x \cdot (y \cdot z) = (x \cdot y) \cdot z$
  \item $\forall x,y: x \cdot y = y \cdot x$
  \item $\forall x: x \cdot 1 = x$
  \item $\forall x: x = 0 \vee (\exists y: x \cdot y = 1)$.
  \end{itemize}
  
  It is immediate that $\fieldthy$ is a $\ringsig$-theory, and that a
  $\ringsig$-structure $\mathcal{M}$ is a field if and only if
  $\mathcal{M} \models \fieldthy$.
\end{example}

\begin{example}
  The \defterm{theory of algebraically closed fields}, $\acf$, is the
  set of consequences of the sentences listed in
  Example~\ref{example:fieldthy}, together with an assertion, for each
  $d > 1$, that every polynomial of degree $d$ has a root. For
  example, for $d = 2$, such a sentence might be $$\forall \tuple{a}:
  \exists x: x^2 + a_1x + a_0 = 0 \text{.}$$ Note that it is often
  convenient, as it is here, to use extensive shorthand when writing
  sentences and formulas when no ambiguity problems arise (e.g., $x^2$
  in place of $x \cdot x$, $t_1 + t_2 + t_3$ in place of $t_1 + (t_2 +
  t_3)$).

  As in Example~\ref{example:fieldthy}, a $\ringsig$-structure is an
  algebraically closed field if and only if it models $\acf$.
\end{example}

\subsection{Formula Equivalence and Quantifier Depth}

Two formulas may express the same idea, even though they are written
differently. This equivalence may be absolute, or it may depend on the
underlying theory (e.g., it may rely on a particular symbol being
associative). It is particularly useful to know that every formula is
equivalent (in the absolute sense) to one where all the quantifiers
are in the front, since we can classify these formulas by the patterns
of quantifiers they contain.

\begin{defn}
  Let $\sigma$ be a signature, and let $\phi(\tuple{x})$ and
  $\psi(\tuple{x})$ be $\sigma$-formulas in $\tuple{x}$, where
  $\tuple{x}$ has length $n$. Say $\phi(\tuple{x})$ is
  \defterm{absolutely equivalent} to $\psi(\tuple{x})$, and write
  $\phi(\tuple{x}) \sim \psi(\tuple{x})$, if for all
  $\sigma$-structures $\mathcal{M}$, $\phi^\mathcal{M} =
  \psi^\mathcal{M}$.

  If $T$ is a $\sigma$-theory, say $\phi(\tuple{x})$ is
  \defterm{equivalent modulo $T$} to $\psi(\tuple{x})$, and write
  $\phi(\tuple{x}) \sim_T \psi(\tuple{x})$, if for all
  $\sigma$-structures $\mathcal{M}$ with $\mathcal{M} \models T$,
  $\phi^\mathcal{M} = \psi^\mathcal{M}$.
\end{defn}

\begin{defn}\label{defn:quantifier-depth}
  Fix a $\sigma$-theory $T$. Say that a $\sigma$-formula is
  \defterm{quantifier free} if it does not contain any
  quantifiers. Say that a formula is $\Sigma_0$ and $\Pi_0$ modulo $T$
  if it is equivalent modulo $T$ to a quantifier free formula.

  Say that a formula $\phi(\tuple{x})$ is $\Sigma_{n+1}$ (modulo $T$)
  if it is equivalent modulo $T$ to $\exists \tuple{y} :
  \psi(\tuple{x}, \tuple{y})$ for some $\Pi_n$ (modulo $T$) formula
  $\psi(\tuple{x}, \tuple{y})$. Similarly, say that $\phi(\tuple{x})$
  is $\Pi_{n+1}$ if it is equivalent to $\forall \tuple{y} :
  \psi(\tuple{x}, \tuple{y})$ for some $\Sigma_n$ formula $T$.

  Say that a formula is $\Delta_n$ if it is both $\Sigma_n$ and
  $\Pi_n$.

  In the previous definitions, if we require absolute equivalence
  rather than equivalence modulo $T$, we say that a formula is
  \defterm{absolutely} $\Sigma_n$, $\Pi_n$, or $\Delta_n$.
\end{defn}

The ideas in Definition~\ref{defn:quantifier-depth} allow us to
describe the alternations of quantifiers needed to express a
concept. Informally, throughout mathematics, these quantifier
alternations measure the complexity of an idea. For example, one
perspective on why calculus is conceptually challenging for students
is that limits are a $\Pi_3$ definition (for all $\epsilon$, there is
a $\delta$ such that for all $x$ \ldots), and calculus is typically
one's first exposure to definitions with three quantifier
alternations. In both model theory and computability, the ability or
inability to express a definition at some level of this hierarchy is a
useful dividing line.

\begin{fact}
  The following are some basic, useful facts about quantifier depth.
  \begin{enumerate}
  \item Every formula is absolutely $\Sigma_n$ or $\Pi_n$ for some
    $n$, since it is possible to rewrite formulas to bring all
    quantifiers to the front.
  \item If a formula is either $\Pi_n$ or $\Sigma_n$, it is
    $\Delta_{n+1}$, since it is always possible to add quantifiers
    over extra, unused variables.
  \item A formula is $\Sigma_n$ if and only if its negation is
    $\Pi_n$. A formula is $\Delta_n$ if and only if its negation is
    $\Delta_n$.
  \item Let $\phi(\tuple{x}, \tuple{y})$ be such that, in every
    $\mathcal{M} \models T$, $\phi^\mathcal{M}$ is the graph of a
    (total) function. Then $\phi(\tuple{x},\tuple{y})$ is $\Sigma_n$
    (modulo $T$) if and only if it is $\Delta_n$ (modulo $T$).
  \end{enumerate}
\end{fact}

\end{document}
