\documentclass{article}
\usepackage{amsmath,amsthm,amssymb}
\usepackage{hyperref}

\theoremstyle{plain}
\newtheorem{thm}{Theorem}[section]
\newtheorem{prop}[thm]{Proposition}
\newtheorem{lemma}[thm]{Lemma}

\theoremstyle{definition}
\newtheorem{defn}[thm]{Definition}
\newtheorem{fact}[thm]{Fact}

\title{A Brief Taste of Quantifiers}
\author{Mike Haskel}
\date{August 22, 2015}


\begin{document}
\maketitle

\begin{abstract}
  These are notes for a talk given on August 22, 2015 as part of the
  University of Notre Dame's orientation for incoming math graduate
  students. The purpose of the talk is to introduce the new students
  to the material covered in the first year logic course, which
  primarily covers both model theory and computability. The unifying
  theme of the talk is first-order definability and quantifier
  complexity. In model theory, we discuss quantifier elimination in
  algebraically closed fields and prove the Ax-Grothendieck
  Theorem. In computability, we introduce the arithmetical hierarchy,
  discuss its relationship with computability, and prove that it does
  not collapse.
\end{abstract}

\end{document}
