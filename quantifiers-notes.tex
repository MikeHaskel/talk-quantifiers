\documentclass{article}
\usepackage{amsmath,amsthm,amssymb}
\usepackage{hyperref}
\usepackage{biblatex}

\addbibresource{quantifiers.bib}

\theoremstyle{plain}
\newtheorem{thm}{Theorem}[section]
\newtheorem{prop}[thm]{Proposition}
\newtheorem{lemma}[thm]{Lemma}

\theoremstyle{definition}
\newtheorem{defn}[thm]{Definition}
\newtheorem{fact}[thm]{Fact}
\newtheorem{example}[thm]{Example}

\newcommand{\defterm}{\emph}
\newcommand{\ringsig}{\sigma_{\operatorname{rng}}}
\newcommand{\arithsig}{\sigma_{\operatorname{arith}}}
\newcommand{\fieldthy}{T_{\operatorname{fld}}}
\newcommand{\acf}{T_{\operatorname{acf}}}
\newcommand{\tuple}{\bar}
\DeclareMathOperator{\Cn}{Cn}
\DeclareMathOperator{\PR}{PR}
\DeclareMathOperator{\TA}{TA}
\newcommand{\tathy}{T_{\TA}}
\newcommand{\Rho}{\mathrm{P}}
\renewcommand{\phi}{\varphi}
\newcommand{\lgnum}{\ulcorner}
\newcommand{\rgnum}{\urcorner}


\title{A Brief Taste of Quantifiers}
\author{Mike Haskel}
\date{August 22, 2015}


\begin{document}
\maketitle

\begin{abstract}
  These are notes for a talk given on August 22, 2015 as part of the
  University of Notre Dame's orientation for incoming math graduate
  students. The purpose of the talk is to introduce the new students
  to the material covered in the first year logic course, which
  primarily covers both model theory and computability. The unifying
  theme of the talk is first-order definability and quantifier
  complexity. In model theory, we discuss quantifier elimination in
  algebraically closed fields and prove the Ax-Grothendieck
  Theorem. In computability, we introduce the arithmetical hierarchy,
  discuss its relationship with computability in general, and prove
  that it does not collapse.
\end{abstract}

\section{Introduction}

These notes introduce some basic definitions from computability and
model theory, and give brief examples of their use. They should be
accessible to beginning graduate students in mathematics, even without
any background specifically in mathematical logic. While model theory
and computability have a common history, contemporary work in these
fields is often quite unrelated. For the purpose of general coherence,
these notes present a common approach in terms of quantifier
complexity. While slightly contrived, this common framework is useful
in that it enables a discussion of the relationship between the two
fields.

\section{Basic Framework and Notation}

This section contains classical definitions that are most closely
associated with model theory. For a more in-depth treatment, see
\cite{marker}.

\subsection{Signatures, Structures, and Formulas}

\begin{defn}
  A \defterm{signature} $\sigma$ is a collection of constant symbols,
  function symbols, and relation symbols, together with an
  \defterm{arity} for each function and relation symbol (that is, a
  natural number specifying how many arguments the symbol takes).
\end{defn}

\begin{defn}
  Given a signature $\sigma$, a \defterm{$\sigma$-structure} is a set
  and an interpretation for every symbol in $\sigma$. That is, a
  $\sigma$-structure $\mathcal{M}$ is a set $M$ together with
  \begin{itemize}
  \item an element $c^\mathcal{M} \in M$ for every constant symbol $c$,
  \item a function $f^\mathcal{M} : M^n \to M$ for every $n$-ary function symbol $f$, and
  \item a relation $R^\mathcal{M} \subseteq M^n$ for every $n$-ary relation symbol $R$.
  \end{itemize}
\end{defn}

As a notational convention, if $\mathcal{M}$ is a structure, $M$ will
always denote its underlying set.

\begin{example}
  The \defterm{signature of rings}, $\ringsig$, consists of constant
  symbols $0, 1$, binary operations $+, \cdot$, and a unary operation
  $-$. Every ring is naturally a $\ringsig$-structure.
\end{example}

Note that being a $\ringsig$-structure is actually much weaker than
being a ring: a $\ringsig$-structure need not satisfy any of the ring
axioms. To actually study rings with this framework, for example, we
need a little more.

\begin{defn}
  Given a signature $\sigma$ and a tuple of variable symbols
  $\tuple{x}$, a \defterm{$\sigma$-formula in $\tuple{x}$},
  $\phi(\tuple{x})$, is a sequence of symbols representing a
  first-order claim, with free variables from among $\tuple{x}$ and
  primitive symbols from $\sigma$.

  Terms in a formula are built as constant symbols, variable symbols,
  and application of function symbols to other terms (respecting
  arity, of course). The formulas themselves are built as $t_1 = t_2$
  (where $t_1, t_2$ are terms), application of relation symbols to
  terms (respecting arity), finite conjunctions, finite disjunctions,
  negations, implications, and universal and existential quantifiers.

  A \defterm{sentence} $\rho$ is a formula with no free variables.
\end{defn}

\begin{defn}
  Let $\sigma$ be a signature, $\mathcal{M}$ a $\sigma$-structure,
  $\phi(\tuple{x})$ a $\sigma$-formula, and $\tuple{a} \in M^n$ (where
  $\tuple{a}, \tuple{x}$ both have length $n$). Then write
  $\mathcal{M} \models \phi(\tuple{a})$ if the formula
  $\phi(\tuple{x})$ holds of $\tuple{a}$ in the obvious way, with
  constant, function, and relation symbols interpreted via
  $\mathcal{M}$, and quantifiers ranging over $M$. Write
  $\phi^\mathcal{M} = \{ \tuple{a} \in M^n \mid \mathcal{M} \models
  \phi(\tuple{a})\}$, the set defined in $\mathcal{M}$ by
  $\phi(\tuple{x})$.

  The double turnstyle symbol, $\models$, is pronounced ``models.''
\end{defn}

\begin{example}
  Let $\phi(x)$ be $\exists y: x \cdot y = 1$, a
  $\ringsig$-formula. If $\mathcal{M}$ is a ring and $a \in M$, the
  $\mathcal{M} \models \phi(a)$ if and only if $a$ has a
  multiplicative inverse.
\end{example}

\begin{example}
  Let $\rho$ be $\forall x: x = 0 \vee (\exists y: x \cdot y = 1)$, a
  $\ringsig$-sentence. If $\mathcal{M}$ is a ring, then $\mathcal{M}
  \models \rho$ if and only if $\mathcal{M}$ is furthermore a field.
\end{example}

\begin{defn}
  Let $\sigma$ be a signature and $\Rho$ set of $\sigma$-sentences.
  \begin{itemize}
  \item If $\mathcal{M}$ is a $\sigma$-structure, write $\mathcal{M}
    \models \Rho$ if $\mathcal{M} \models \rho$ for every $\rho \in
    \Rho$.
  \item If $\mathcal{\rho}$ is a $\sigma$-sentence (not necessarily in
    $\Rho$), write $\Rho \models \rho$ if, for all $\sigma$-structures
    $\mathcal{M}$ with $\mathcal{M} \models \Rho$, $\mathcal{M}
    \models \rho$.
  \end{itemize}
\end{defn}

\begin{defn}
  Let $\sigma$ be a signature and $\Rho$ a set of $\sigma$-sentences.
  \begin{itemize}
  \item Define $\Cn(\Rho)$, the \defterm{consequences} of $\Rho$, to
    be $$\{\rho \mid \Rho \models \rho\}\text{.}$$
  \item Say that $\Rho$ is \defterm{consistent} if it has a model,
    i.e., there is $\mathcal{M}$ such that $\mathcal{M} \models \Rho$.
  \item Say that $\Rho$ is a $\sigma$-\defterm{theory} if it is closed
    under consequence (i.e.\ $\Cn(\Rho) = \Rho$) and consistent.
  \end{itemize}
\end{defn}

By convention, theories are usually written $T$ instead of $\Rho$.

\begin{example}\label{example:fieldthy}
  The \defterm{theory of fields}, $\fieldthy$, is the set of consequences of
  the following $\ringsig$-sentences:
  \begin{itemize}
  \item $\forall x,y,z: x + (y + z) = (x + y) + z$
  \item $\forall x,y: x + y = y + x$
  \item $\forall x: x + 0 = x$
  \item $\forall x: x + (-x) = 0$
  \item $\forall x,y,z: x \cdot (y \cdot z) = (x \cdot y) \cdot z$
  \item $\forall x,y: x \cdot y = y \cdot x$
  \item $\forall x: x \cdot 1 = x$
  \item $\forall x: x = 0 \vee (\exists y: x \cdot y = 1)$.
  \end{itemize}
  
  It is immediate that $\fieldthy$ is a $\ringsig$-theory, and that a
  $\ringsig$-structure $\mathcal{M}$ is a field if and only if
  $\mathcal{M} \models \fieldthy$.
\end{example}

\begin{example}
  The \defterm{theory of algebraically closed fields}, $\acf$, is the
  set of consequences of the sentences listed in
  Example~\ref{example:fieldthy}, together with an assertion, for each
  $d > 1$, that every polynomial of degree $d$ has a root. For
  example, for $d = 2$, such a sentence might be $$\forall \tuple{a}:
  \exists x: x^2 + a_1x + a_0 = 0 \text{.}$$ Note that it is often
  convenient, as it is here, to use extensive shorthand when writing
  sentences and formulas when no ambiguity problems arise (e.g., $x^2$
  in place of $x \cdot x$, $t_1 + t_2 + t_3$ in place of $t_1 + (t_2 +
  t_3)$).

  As in Example~\ref{example:fieldthy}, a $\ringsig$-structure is an
  algebraically closed field if and only if it models $\acf$.
\end{example}

\subsection{Formula Equivalence and Quantifier Depth}

Two formulas may express the same idea, even though they are written
differently. This equivalence may be absolute, or it may depend on the
underlying theory (e.g., it may rely on a particular symbol being
associative). It is particularly useful to know that every formula is
equivalent (in the absolute sense) to one where all the quantifiers
are in the front, since we can classify these formulas by the patterns
of quantifiers they contain.

\begin{defn}
  Let $\sigma$ be a signature, and let $\phi(\tuple{x})$ and
  $\psi(\tuple{x})$ be $\sigma$-formulas in $\tuple{x}$, where
  $\tuple{x}$ has length $n$. Say $\phi(\tuple{x})$ is
  \defterm{absolutely equivalent} to $\psi(\tuple{x})$, and write
  $\phi(\tuple{x}) \sim \psi(\tuple{x})$, if for all
  $\sigma$-structures $\mathcal{M}$, $\phi^\mathcal{M} =
  \psi^\mathcal{M}$.

  If $T$ is a $\sigma$-theory, say $\phi(\tuple{x})$ is
  \defterm{equivalent modulo $T$} to $\psi(\tuple{x})$, and write
  $\phi(\tuple{x}) \sim_T \psi(\tuple{x})$, if for all
  $\sigma$-structures $\mathcal{M}$ with $\mathcal{M} \models T$,
  $\phi^\mathcal{M} = \psi^\mathcal{M}$.
\end{defn}

\begin{defn}\label{defn:quantifier-depth}
  Fix a $\sigma$-theory $T$. Say that a $\sigma$-formula is
  \defterm{quantifier free} if it does not contain any
  quantifiers. Say that a formula is $\Sigma_0$ and $\Pi_0$ modulo $T$
  if it is equivalent modulo $T$ to a quantifier free formula.

  Say that a formula $\phi(\tuple{x})$ is $\Sigma_{n+1}$ (modulo $T$)
  if it is equivalent modulo $T$ to $\exists \tuple{y} :
  \psi(\tuple{x}, \tuple{y})$ for some $\Pi_n$ (modulo $T$) formula
  $\psi(\tuple{x}, \tuple{y})$. Similarly, say that $\phi(\tuple{x})$
  is $\Pi_{n+1}$ if it is equivalent to $\forall \tuple{y} :
  \psi(\tuple{x}, \tuple{y})$ for some $\Sigma_n$ formula $T$.

  Say that a formula is $\Delta_n$ if it is both $\Sigma_n$ and
  $\Pi_n$.

  In the previous definitions, if we require absolute equivalence
  rather than equivalence modulo $T$, we say that a formula is
  \defterm{absolutely} $\Sigma_n$, $\Pi_n$, or $\Delta_n$.
\end{defn}

The ideas in Definition~\ref{defn:quantifier-depth} allow us to
describe the alternations of quantifiers needed to express a
concept. Informally, throughout mathematics, these quantifier
alternations measure the complexity of an idea. For example, one
perspective on why calculus is conceptually challenging for students
is that limits are a $\Pi_3$ definition (for all $\epsilon$, there is
a $\delta$ such that for all $x$ \ldots), and calculus is typically
one's first exposure to definitions with three quantifier
alternations. In both model theory and computability, the ability or
inability to express a definition at some level of this hierarchy is a
useful dividing line.

\begin{fact}\label{fact:qfhier-basic}
  The following are some basic, useful facts about quantifier depth.
  \begin{enumerate}
  \item Every formula is absolutely $\Sigma_n$ or $\Pi_n$ for some
    $n$, since it is possible to rewrite formulas to bring all
    quantifiers to the front.
  \item If a formula is either $\Pi_n$ or $\Sigma_n$, it is
    $\Delta_{n+1}$, since it is always possible to add quantifiers
    over extra, unused variables.
  \item A formula is $\Sigma_n$ if and only if its negation is
    $\Pi_n$. A formula is $\Delta_n$ if and only if its negation is
    $\Delta_n$.
  \item Let $\phi(\tuple{x}, \tuple{y})$ be such that, in every
    $\mathcal{M} \models T$, $\phi^\mathcal{M}$ is the graph of a
    (total) function. Then $\phi(\tuple{x},\tuple{y})$ is $\Sigma_n$
    (modulo $T$) if and only if it is $\Delta_n$ (modulo $T$).
  \end{enumerate}
\end{fact}

\section{Model Theory}

Model theory, as a general rule, studies theories that are not too
complicated. Its catalog of tools and techniques make use of
situations where a theory is lacking some kind of complexity. An
example of such a lack of complexity if a definition is at some
particular level of the quantifier depth hierarchy. In this section,
we exploit such a situation to prove the classic Ax-Grothendieck
Theorem.

The results in this section are classical. David Marker exposits
quantifier elimination, including quantifier elimination in
algebraically closed fields, in \cite[Chapter~3]{marker}. He proves
the Ax-Grothendieck Theorem (our Theorem~\ref{thm:ax-groth}) as
\cite[Theorem~2.2.11]{marker}, though he takes a different, somewhat
related approach using compactness rather than quantifier
elimination. I don't know of a thorough written treatment of the
quantifier elimination approach. I learned of it from Anand Pillay,
who gives a very brief presentation of the idea in his online
notes\cite[Corollaries 3.8 and 3.9]{anandnotes}.

\begin{thm}[Ax-Grothendieck]\label{thm:ax-groth}
  Let $p : \mathbb{C}^n \to \mathbb{C}^n$ be an injective polynomial
  map. Then $p$ is surjective.
\end{thm}

Note that Theorem~\ref{thm:ax-groth} is not about complex analysis: it
is in particular false if we relax ``polynomial'' to ``holomorphic.''
The fact that polynomial maps are definable as formulas is essential,
as it allows us to use facts from model theory. We show
Theorem~\ref{thm:ax-groth} as a corollary to the more model-theoretic
Theorem~\ref{thm:ax-groth-modeltheory}.

\begin{thm}\label{thm:ax-groth-modeltheory}
  Let $\rho$ be a $\ringsig$-sentence that is $\Pi_2$ modulo
  $\fieldthy$ and true in every finite field. Then $\rho$ is true in
  every algebraically closed field.
\end{thm}

We devote the rest of this section to establishing these theorems. The
main steps are:
\begin{enumerate}
\item show that any sentence true in an algebraically closed field is
  true in some $\overline{\mathbb{F}_p}$, and
\item show that any $\Pi_2$ sentence that fails in
  $\overline{\mathbb{F}_p}$ fails in a finite subfield.
\end{enumerate}

Notre Dame's basic logic course will establish a key fact: that $\acf$
eliminates quantifiers. This is a powerful result that is nevertheless
follows easily from combining the basic tools of both model theory and
field theory. In these notes, however, we will treat the result as a
black box.

\begin{defn}
  A theory $T$ \defterm{eliminates quantifiers} (or \defterm{has
    quantifier elimination}) if every formula is $\Delta_0$ modulo
  $T$. That is, for every formula $\phi(\tuple{x})$, there is a
  quantifier free formula $\psi(\tuple{x})$ such that $\phi(\tuple{x})
  \sim_T \psi(\tuple{x})$.
\end{defn}

\begin{fact}\label{fact:acf-qe}
  $\acf$ eliminates quantifiers.
\end{fact}

\begin{prop}\label{prop:anything-goestofinitecharacteristic}
  Let $\rho$ be a $\ringsig$-sentence. Then, if $\rho$ holds in some
  algebraically closed field, it holds in some
  $\overline{\mathbb{F}_p}$.
\end{prop}
\begin{proof}
  Since a sentence is just a formula with no free variables,
  Fact~\ref{fact:acf-qe} yields a quantifier free sentence $\rho'$
  that is equivalent modulo $\acf$ to $\rho$. Such quantifier free
  sentences are quite inexpressive, however: they are (up to
  equivalence modulo $\fieldthy$, even) just boolean combinations of
  equalities of the form $n = 0$. Consequently, whether $\rho$ holds
  in an algebrically closed field depends only on the field's
  characteristic. Furthermore, since a single sentence can only
  contain finitely many equalities of the form $n = 0$, $\rho$ is true
  in characteristic 0 if and only if it is true in some large
  characteristic $p$ (by taking $p$ greater than every $n$ mentioned).

  We therefore have that, if $\rho$ holds in an algebraically closed
  field, it holds in all algebraically closed fields of some finite
  characteristic $p$, and in particular holds in
  $\overline{\mathbb{F}_p}$.
\end{proof}

\begin{prop}\label{prop:sig2-goestofinitefield}
  Let $\rho$ be a $\ringsig$ sentence that is $\Sigma_2$ modulo
  $\fieldthy$, and assume $\overline{\mathbb{F}_p} \models \rho$. Then
  there is a finite subfield $k$ of $\overline{\mathbb{F}_p}$ such
  that $k \models \rho$.
\end{prop}
\begin{proof}
  Without loss of generality, assume $\rho$ has the form $\exists
  \tuple{x} : \forall \tuple{y} : \phi(\tuple{x},\tuple{y})$, where
  $\phi(\tuple{x},\tuple{y})$ is quantifier free. Then we can take
  $\tuple{a} \in \overline{\mathbb{F}_p}$ such that
  $\overline{\mathbb{F}_p} \models \forall \tuple{y}:
  \phi(\tuple{a},\tuple{y})$. Let $k = \mathbb{F}_p(\tuple{a})$, a
  finite subfield of $\overline{\mathbb{F}_p}$. We wish to show that
  $k \models \forall y: \phi(\tuple{a},\tuple{y})$, so let $\tuple{b}
  \in k$ be arbitrary. We know that $\overline{\mathbb{F}_p} \models
  \phi(\tuple{a},\tuple{b})$. Since $\phi(\tuple{x},\tuple{y})$ is
  quantifier free, its truth does not depend on the ambient field (it
  is just a boolean combination of polynomial equalities), so we can
  conclude $k \models \phi(\tuple{a},\tuple{b})$ as desired.
\end{proof}

\begin{proof}[Proof of Theorem~\ref{thm:ax-groth-modeltheory}]
  Let $\rho$ be $\Pi_2$ modulo $\fieldthy$. We will show the theorem
  by contrapositive, i.e., if $\rho$ fails in some algebraically
  closed field, it fails in some finite field.

  Note that, since $\rho$ is $\Pi_2$ modulo $\fieldthy$, its negation,
  $\neg \rho$, is $\Sigma_2$ modulo $\fieldthy$. Since $\neg \rho$
  holds exactly where $\rho$ fails, the result follows from
  Propositions \ref{prop:anything-goestofinitecharacteristic} and
  \ref{prop:sig2-goestofinitefield}.
\end{proof}

\begin{proof}[Proof of Theorem~\ref{thm:ax-groth}]
  We use Theorem~\ref{thm:ax-groth-modeltheory}, together with the
  realizations that we can phrase the claim as a collection of
  absolutely $\Pi_2$ sentences and that any injective map from a
  finite set to itself is surjective.

  Given a degree $d$ and dimension $n$, we can form a sentence
  $\rho_{n,d}$ claiming that every injective, degree $d$ polynomial
  map from $n$ values to $n$ values is surjective. Expanding somewhat,
  $\rho_{n,d}$ says that, for every degree $d$ polynomial map from $n$
  values to $n$ values, either
  \begin{enumerate}
  \item for all $\tuple{y}$, there is $\tuple{x}$ such that $p(\tuple{x}) = \tuple{y}$, or
  \item there are $\tuple{x}$ and $\tuple{y}$ such that $\tuple{x}
    \neq \tuple{y}$ and $p(\tuple{x}) = p(\tuple{y})$.
  \end{enumerate}
  Since we can quantify over polynomial maps (of fixed degree) by
  quantifying over the parameters, we can clearly form such a
  sentence. Furthermore, one can inspect the preceding description of
  $\rho_{n,d}$ to see that $\rho_{n,d}$ is absolutely $\Pi_2$. By our
  earlier remarks, we have thus shown the theorem.
\end{proof}

\section{Computability}

If model theory studies theories theories and structures lacking
complexity, computability studies one structure in particular that has
a lot of complexicy. This structure, \defterm{true arithmetic}, and
its theory provide the right context for studying the extent to which
certain sets and functions are computable by abstract mechanical
processes. Quantifier depth is again important as the basis for the
\defterm{arithmetical hierarchy}.

Again, the results in this section are classical. Robert Soare
presents these ideas in depth in \cite{soare}.

Defining the true arithmetic structure, and even its signature, is
more difficult than in the other examples discussed here. We first
need to understand the primitive recursive functions.

\begin{defn}\label{defn:primrec}
  For any $n \in \mathbb{N}$, $\PR_n$ is a particular family of
  functions $\mathbb{N}^n \to \mathbb{N}$. A function in any one of
  these families is called \defterm{primitive recursive}. These
  families are by definition the smallest needed so that:
  \begin{itemize}
  \item for all $n$ and $i \leq n$, $\left(\tuple{x} \mapsto x_i\right) \in \PR_n$;
  \item $\left(() \mapsto 0\right) \in \PR_0$ (this is a ``0-ary''
    function that takes an empty tuple as input and returns 0);
  \item $\left(x \mapsto x + 1\right) \in \PR_1$;
  \item for all $n$ and $m$, if $g \in \PR_n$, $f_1,\ldots,f_n \in
    \PR_m$, and $h : \mathbb{N}^m \to \mathbb{N}$ is defined
    by $$h(\tuple{x}) =
    g(f_1(\tuple{x}),\ldots,f_n(\tuple{x}))\text{,}$$ then $h \in
    \PR_m$; and
  \item for all $n$, if $z \in \PR_n$, $s \in \PR_{n+2}$, and $f :
    \mathbb{N}^{n+1} \to \mathbb{N}$ is defined by
    \begin{align*}
      f(0,\tuple{y}) &= z(\tuple{y}) \\
      f(x+1,\tuple{y}) &= s(x,f(x,\tuple{y}),\tuple{y}) \text{,}
    \end{align*}
    then $f \in \PR_{n+1}$.
  \end{itemize}

  A \defterm{primitive recursive set} is a subset of $\mathbb{N}^n$
  whose characteristic function is in $\PR_n$.
\end{defn}

In essence, the primitive recursive functions are those those built by
combining basic operations and by repeated iteration, where the number
of iterations is known ahead of time. All the basic number theory
functions are primitive recursive, including addition,
muplitplication, exponentiation, primality testing, finding the $n$th
prime, and many more. It is clear, also, from inspecting the
definition, that any primitive recursive function is computable by a
mechanical process (i.e., an algorithm) that eventually halts and
gives the right answer.

\begin{defn}
  The \defterm{signature of arithmetic}, $\arithsig$, has every
  natural number as a constant symbol, every primitive recursive
  function as a function symbol (of the appropriate arity), and every
  primitive recursive set as a relation symbol (of the appropriate
  arity). \defterm{True arithemetic}, $\TA$, is the
  $\arithsig$-structure whose underlying set is $\mathbb{N}$, with all
  symbols interpreted in the natural way. The \defterm{theory of true
    arithmetic}, $\tathy$, is the set of all $\arithsig$-sentences
  true in $\TA$.
\end{defn}

We will often use equivalence of formulas modulo $\tathy$. Note,
however, that two formulas are equivalent modulo $\tathy$ if and only
if they define the same set in $\TA$, since
\begin{align*}
  \phi(\tuple{x}) \sim_{\tathy} \psi(\tuple{x})
  &\iff \phi^\mathcal{M} = \psi^\mathcal{M} \text{ in all } \mathcal{M} \models \tathy \\
  &\iff \tathy \models \forall \tuple{x}: \phi(\tuple{x}) \leftrightarrow \psi(\tuple{x}) \\
  &\iff \TA \models \forall \tuple{x}: \phi(\tuple{x}) \leftrightarrow \psi(\tuple{x}) \\
  &\iff \phi^{\TA} = \psi^{\TA} \text{.}
\end{align*}

For the rest of this section, we only consider sentences and formulas
in $\arithsig$, and in any discussion of formula equivalence or being
$\Sigma_n$, $\Pi_n$, or $\Delta_n$ we mean modulo
$\tathy$. Furthermore, we say that a subset of $\mathbb{N}^n$ is
$\Sigma_n$ if it is defined by a $\Sigma_n$ formula, etc., and that a
function is $\Sigma_n$, etc., if its graph is.

There is one final basic fact which simplifies reasoning in some
cases.

\begin{fact}\label{fact:arith-simpler-normalform}
  Every formula $\phi(\tuple{x})$ is equivalent either to one of the
  form
  $$\exists y_1 \forall y_2 \exists y_3 \ldots R(\tuple{x},\tuple{y})$$
  or to one of the form
  $$\forall y_1 \exists y_2 \forall y_3 \ldots R(\tuple{x},\tuple{y})\text{,}$$
  where in either case $R$ is a relation symbol (i.e., a primitive
  recursive set). Furthermore, considering only formulas of this form
  does not affect the $\Sigma_n$, $\Pi_n$, and $\Delta_n$ hierarchy.
\end{fact}

Fact~\ref{fact:arith-simpler-normalform} provides two simplifications:
\begin{enumerate}
\item we need only consider quantifiers over a single variable, rather
  than tuples of variables, at any level of the hierarchy; and
\item rather than consider arbitrary quantifier free formulas as a
  base case, we need only consider a single primitive recursive set.
\end{enumerate}
These simplifications are justified essentially by the expressivity of
the primitive recursive sets themselves: they can decode multiple
paramters from a single value (the first simplification), and the
boolean combination operations in a quantifier free formula are
themselves primitive recursive (the second simplification).

\subsection{The Arithmetical Hierarchy and Computability}

The classification of sets (and functions) in $\TA$ into $\Sigma_n$,
$\Pi_n$, and $\Delta_n$ forms the \defterm{arithmetical hierarchy},
and any set (or function) with such a classification (i.e., definable
by some $\arithsig$-formula) is called \defterm{arithmetical}. The
zeroth level, $\Sigma_0 = \Pi_0 = \Delta_0$, comprises the primitive
recursive sets. The classifications of the first level also have
special status in terms of computability.

\begin{defn}
  A set is \defterm{computably enumerable}, or \defterm{c.e.}, if it
  is $\Sigma_1$. A set is \defterm{co-computably enumerable}, or
  \defterm{co-c.e.}, if it is $\Pi_1$ (i.e., its complement is
  c.e.). A set is \defterm{computable} if it is $\Delta_1$.
\end{defn}

To justify this special status, consider what it means to be a
$\Sigma_1$ set. A $\Sigma_1$ set is defined by some formula of the
form $$\phi(\tuple{x}) = \exists y : R(\tuple{x},y)\text{,}$$ with $R$
primitive recursive. To check whether $\phi(\tuple{x})$ holds of
$\tuple{a}$, we can simply start trying possible values for $y$. That
is, we first check whether $R(\tuple{a},0)$ holds, then check
$R(\tuple{a},1)$ holds, etc. If and only if $\phi(\tuple{a})$ holds,
this process will eventually halt and provide confirmation of that
fact. For any c.e.\ set, then, there is a mechanical process for
verifying that something is a member (verifying \emph{non}-membership
may be impossible, however). On the other hand, for any co-c.e.\ set,
there is a process for verifying non-membership, but not necessarily
for verifying membership. For a computable set, we can determine
whether any tuple is or is not a member by running the processes to
verify membership and non-membership in parallel. Eventually, one will
halt and tell us whether our candidate is a member.

We have established that c.e.\ sets are special in that there exists a
mechanical process for verifying membership. Such processes
furthermore \emph{characterize} the c.e.\ sets (and consequently, the
preceding descriptions characterize the co-c.e.\ and computable
sets). This fact follows from the expressivity of the primitive
recursive sets and functions: given any description of a mechanical
process to verify membership in a set, following that process consists
of taking a finite sequence of steps. For any instance of following
the algorithm, it is possible to encode the entire sequence of steps
we took, along with the information we had at each step, into a single
number. We can furthermore encode this information so that a function
to verify that the algorithm was performed correctly is primitive
recursive. If membership in a set $X$ is verifiable by a mechanical
process, we can therefore make a formula defining $X$ along the lines
of: ``there is a number $y$ encoding steps to correctly show that
$\tuple{x} \in X$, per the given algorithm.''

\begin{fact}[Church-Turing Thesis]\label{fact:ct-thesis}
  A set is c.e. if and only if there is a mechanical process to verify
  membership in the set. A set is co-c.e. if and only if there is a
  mechanical process to verify non-membership in it. A set is
  computable if and only if there is a mechanical process to decide
  membership in it.
\end{fact}

Fact~\ref{fact:ct-thesis} is intentionally imprecise. There are many
different approaches to formalizing the intuition for what constitutes
a mechanical process, and Fact~\ref{fact:ct-thesis} expresses that
every different way mathematicians have tried to do so results in
something equivalent to the definitions given here for c.e., co-c.e.,
and computable sets.

More generally, Post's theorem, which we do not present here in
detail, says that a set is $\Sigma_{n+1}$ if and only if there is a
mechanical process to verify membership, given that we allow such a
process access to an ``oracle'' that can correctly decide membership
in any $\Sigma_n$ set (or equivalently, in any $\Pi_n$ set). The
entire arithemetical hierarchy is thus deeply connected to
computability.

\subsection{Universal Sets and Non-Collapse}

In algebraically closed fields, quantifier elimation means that all
definable sets are in fact $\Delta_0$. In contrast, in the
arithmetical hierarchy, we see new sets at every level (it does not
``collapse''). In particular, for every $n \geq 0$, there is a set
that is $\Sigma_n$ but not $\Pi_n$ (and therefore not $\Sigma_m$,
$\Pi_m$, or $\Delta_m$ for any $m < n$, by
Fact~\ref{fact:qfhier-basic}). We form these sets as particular
\defterm{universal sets}, which are useful in their own right.

\begin{defn}
  Let $\mathcal{F}$ be a family of subsets of $\mathbb{N}^n$. A
  \defterm{universal set} for $\mathcal{F}$ is a set $U_\mathcal{F}
  \subseteq \mathbb{N}^{n+1}$ such that, for every $X \in
  \mathcal{F}$, there is a number $\lgnum X \rgnum$ such that
  $\tuple{a} \in X$ if and only if $(\lgnum X \rgnum, \tuple{a}) \in
  U_\mathcal{F}$. That is, every set $X$ in the family has a
  $U_\mathcal{F}$-code, $\lgnum X \rgnum$, and we can recover $X$ by
  supplying $\lgnum X \rgnum$ as the first argument to
  $U_\mathcal{F}$.
\end{defn}

It is clear that any countable family of sets (including each
$\Sigma_n$, $\Pi_n$, and $\Delta_n$) has a universal set. The
interesting question, however, is where such a universal set may lie
in the arithmetical hierarchy.

\begin{prop}\label{prop:univ-sign-not-pin}
  For any $n$ and any $d \geq 1$, assume $U \subseteq
  \mathbb{N}^{d+1}$ is a universal set for the family of $\Sigma_n$
  subsets of $\mathbb{N}^d$. Then $U$ is not $\Pi_n$.
\end{prop}
\begin{proof}
  The heart of this proof is a diagonal argument, similar to the one
  Cantor gave to show there are uncountable many subsets of
  $\mathbb{N}$ and similar to Russell's paradox (the ``set of all sets
  that don't contain themselves''). For clarity, we only show the case
  when $d = 1$. The proof for arbitrary $d \geq 1$ proceeds similarly.

  Assume $U \subseteq \mathbb{N}^2$ is $\Pi_n$. Define $$X = \{e \in
  \mathbb{N} \mid (e,e) \not\in U\}\text{.}$$ Since $U$ is $\Pi_n$,
  its complement is $\Sigma_n$, and therefore $X$ is also
  $\Sigma_n$. However, since $U$ is a universal set for $\Sigma_n$,
  observe:
  \begin{align*}
    \lgnum X \rgnum \in X
    &\iff (\lgnum X \rgnum, \lgnum X \rgnum) \not\in U \text{ (definition of } X \text{)} \\
    &\iff \lgnum X \rgnum \not\in X \text{ (definition of } U \text{),}
  \end{align*}
  a contradiction.
\end{proof}

Though no universal set for $\Sigma_n$ can be $\Pi_n$, we will create
such sets that are $\Sigma_n$ (for $n \geq 1$). These universal sets,
then, provide our examples of $\Sigma_n$ sets that are not $\Pi_n$,
and show that the arithmetical hierarchy does not
collapse. Unfortunately, since $\Sigma_0 = \Pi_0$,
Proposition~\ref{prop:univ-sign-not-pin} shows that no universal
$\Sigma_0$ set can be $\Sigma_0$. We must instead settle for a
$\Delta_1$ set, and we begin our study with that case.

\begin{prop}
  For any $d \geq 0$, there is a $\Delta_1$ set $U_{\Delta_0,d}
  \subseteq \mathbb{N}^{d+1}$ which is universal for the $\Delta_0$
  (i.e., primitive recursive) subsets of $\mathbb{N}^d$.
\end{prop}
\begin{proof}
  Note that any primitive recursive set (or function) is build
  inductively in some way via certain combinations following
  Definition~\ref{defn:primrec}. It is possible to encode any such
  construction in a single number. Given any number $e$ and any
  $\tuple{a} \in \mathbb{N}^d$, it is purely mechanical to figure out
  if $e$ encodes the construction of a primitive recursive function
  and, if it does, follow that construction to evaluate the function
  on $\tuple{a}$. Since, by Fact~\ref{fact:ct-thesis}, a $\Delta_1$
  set is just one for which there is a mechanical process to decide
  membership, we've build a $\Delta_1$ universal set for $\Delta_0$,
  as required.
\end{proof}

\begin{prop}
  For any $d \geq 0$ and $n \geq 1$, there is a $\Sigma_n$ set
  $U_{\Sigma_n,d} \subseteq \mathbb{N}^{d+1}$ which is universal for
  the $\Sigma_n$ subsets of $\mathbb{N}^d$, and a $\Pi_n$ set
  $U_{\Pi_n,d} \subseteq \mathbb{N}^{d+1}$ which is universal for the
  $\Pi_n$ subsets of $\mathbb{N}^d$.
\end{prop}
\begin{proof}
  Proceed by induction on $n$ (with $d$ varying). For the base case,
  set
  \begin{align*}
    U_{\Sigma_1,d}(e,\tuple{x}) &= \exists y : U_{\Delta_0,d+1}(e,\tuple{x},y) \\
    U_{\Pi_1,d}(e,\tuple{x}) &= \forall y : U_{\Delta_0,d+1}(e,\tuple{x},y)
    \text{.}
  \end{align*}
  Since $U_{\Delta_0,d+1}$ is $\Delta_1$, it is clear from the form of
  the definitions that $U_{\Sigma_1,d}$ is $\Sigma_1$ and
  $U_{\Pi_1,d}$ is $\Pi_1$. Furthermore, let $X \subseteq
  \mathbb{N}^d$ be a $\Sigma_1$ set, i.e., defined by $\exists y :
  R(\tuple{x},y)$ for $R$ primitive recursive. Set $\lgnum X \rgnum$,
  the $U_{\Sigma_1,d}$ code for $X$, to be $\lgnum R \rgnum$, the
  $U_{\Delta_0,d+1}$ code for $R$. Then
  \begin{align*}
    \tuple{x} \in X
    &\iff \exists y : R(\tuple{x}, y) \\
    &\iff \exists y : U_{\Delta_0,d+1}(\lgnum X \rgnum, \tuple{x}, y) \\
    &\iff U_{\Sigma_1,d}(\lgnum X \rgnum, \tuple{x})
    \text{,}
  \end{align*}
  So $U_{\Sigma_1,d}$ is a universal function as required. The
  argument for $U_{\Pi_1,d}$ is identical.

  For the inductive case, assume we already have suitable
  $U_{\Sigma_n,d}$ and $U_{\Pi_n,d}$ for all $d$. Then, set
  \begin{align*}
    U_{\Sigma_{n+1},d}(e,\tuple{x}) &= \exists y : U_{\Pi_n,d+1}(e,\tuple{x},y) \\
    U_{\Pi_{n+1},d}(e,\tuple{x}) &= \forall y : U_{\Sigma_n,d+1}(e,\tuple{x},y)
    \text{.}
  \end{align*}
  By reasoning identical to that in the base case, these definitions
  meet our requirements.
\end{proof}

\section{Reflection}

The ideas presented in these notes form a starting point and very
small taste of both model theory and computability.

In model theory, quantifier elimination is one possible criterion for
establishing that a theory is ``low-complexity'' in a meaningful and
useful way. It is possible to weaken this criterion to cover more
theories, such as by requiring all formulas be equivalent to a
$\Sigma_1$ formula, say, instead of a $\Delta_0$ formula. The idea
itself extends to the related \defterm{elimination of imaginaries}, a
criterion that roughly states that all quotients by definable
equivalence relations already exist as definable sets, just as
quantifier elimination states that all definable sets already exists
as sets definable without quantifiers. Other approaches in model
theory include more combinatorial ways of capturing a lack of
complexity, such as \defterm{stability}, \defterm{simplicity}, and
\defterm{NIP}.

In computability theory, a there is a natural extension of the
arithematical hierarchy to a \defterm{hyperarithmetical} hierarchy,
which allows a discussion of sets that are $\Sigma_\alpha$,
$\Pi_\alpha$, and $\Delta_\alpha$ for certain infinite ordinals
$\alpha$.  $\Pi_\alpha$. This approach, however, requires a departure
from the framework of first-order formulas described in these
notes. This hierarchy relates nicely to a certain notion of yet more
expressive sets, the \defterm{analytical} sets. Alongside these
hierarchies, there is another approach to classifying sets of natural
numbers: Turing degrees. Studying these hierarchies and degrees
requires a slew of new tools that go far beyond the diagonal argument
used here.

The author hopes that these notes serve to pique the reader's interest
in mathematical logic and continue further study.

\printbibliography

\end{document}
